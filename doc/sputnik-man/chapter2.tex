\chapter*{\sputnik execution}

%%%%%%%%%%%%%%%%%%%%%%%%%%%%%%%%%%%%%%%%%%%%%%%%%%%%%%%%%%%%%%%%%%%%%%%%%%%%%%%%%%%%

\definecolor{mygreen}{rgb}{0.3, 0.9, 0.1}

%%%%%%%%%%%%%%%%%%%%%%%%%%%%%%%%%%%%%%%%%%%%%%%%%%%%%%%%%%%%%%%%%%%%%%%%%%%%%%%%%%%

\section{Code execution}

The \sputnik programs \macro and \micro can be executed in different modes: standalone
execution or coupled. Generally the standalone execution is used for performing tests of 
each of the codes independently for debugging purposes, while the coupled execution is
used for solving a problem by the multi-scale approach.

\subsection{\macro standalone}

Testing a \macro calculation with input file (see section for further details), this one should not have materials defined with \micflag.
The calculation involves the solution of PDE.
\begin{lstlisting}[frame=single,language=bash]
<@\textcolor{blue}{mpirun}@> -np <np> <@\textcolor{OliveGreen}{macro}@> -input <input.mac>
\end{lstlisting}

\subsection{\micro standalone}

Testing a \micro calculation with input file (see section for further details).
The calculation involves the solution of PDE.
\begin{lstlisting}[frame=single,language=bash]
<@\textcolor{blue}{mpirun}@> -np <np> <@\textcolor{OliveGreen}{micro}@> -input <input.mic>
\end{lstlisting}

\subsection{\macro and \micro coupled}

Performing a multi-scale calculation
\begin{lstlisting}[frame=single,language=bash]
<@\textcolor{blue}{mpirun}@> -np <np_mac> <@\textcolor{OliveGreen}{macro}@> -input <input.mac> :
      -np <np_mic> <@\textcolor{OliveGreen}{micro}@> -input <input.mic>
\end{lstlisting}

%%%%%%%%%%%%%%%%%%%%%%%%%%%%%%%%%%%%%%%%%%%%%%%%%%%%%%%%%%%%%%%%%%%%%%%%%%%%%%%%%%%

\section{\macro input file}

\subsection{Mesh information}

The only input format for the mesh is \texttt{GMSH}.

\begin{lstlisting}[frame=single,language=bash]
<@\textcolor{blue}{\$mesh}@> 
     <@\textcolor{OliveGreen}{mesh_file}@> <mesh_name> 
<@\textcolor{blue}{\$end_mesh}@> 
\end{lstlisting}

\subsection{Boundary Conditions}

\begin{lstlisting}[frame=single,language=bash]
<@\textcolor{blue}{\$boundary}@> 
     <@\textcolor{OliveGreen}{<physical_1>}@> <XXX> order f1 f2 f3
     <@\textcolor{OliveGreen}{<physical_1>}@> <XXX> order f1 f2 f3
     <@\textcolor{OliveGreen}{<physical_1>}@> <XXX> order f1 f2 f3
     <@\textcolor{OliveGreen}{<physical_1>}@> <XXX> order f1 f2 f3
<@\textcolor{blue}{\$end_boundary}@> 
\end{lstlisting}

%%%%%%%%%%%%%%%%%%%%%%%%%%%%%%%%%%%%%%%%%%%%%%%%%%%%%%%%%%%%%%%%%%%%%%%%%%%%%%%%%%%%
\subsection{Schemes}

\begin{lstlisting}[frame=single,language=bash]
<@\textcolor{blue}{\$scheme}@> 
     <@\textcolor{OliveGreen}{scheme}@> <scheme_type>
     <@\textcolor{OliveGreen}{options}@> 
<@\textcolor{blue}{\$end_scheme}@> 
\end{lstlisting}

\texttt{scheme\_type} can be:

\begin{itemize}
\item \texttt{MACRO_MICRO}

Number of micro-structures types. For example an structures having two kind of composite materials
joined together.

\begin{lstlisting}[frame=single]
     <@\textcolor{OliveGreen}{micro_structures}@> #n
\end{lstlisting}

Number of processes for each micro-structure. If \micro is execute with \texttt{np} processes and $\texttt{np}_{\mu} =
\sum_{i} \texttt{proc_{i}}$. Then:

$$\mod{(\texttt{np} ,\texttt{np}_{\mu})} = 0$$

we define the total number of micro-worlds as:

$$\texttt{nw}_{\mu} = \frac{\texttt{np}}{\texttt{np}_{\mu}}$$

\begin{lstlisting}[frame=single]
     <@\textcolor{OliveGreen}{nproc_per_mic}@> #proc_1 #proc_2 ... #proc_n
\end{lstlisting}

$$ \texttt{nproc_mic} = \texttt{np} $$

$$ \texttt{nproc_mic_group} = \texttt{np}_{\mu} $$

$$ \texttt{nmic_worlds} = \texttt{nw}_{\mu} $$

\item \texttt{MACRO_ALONE}   
\item \texttt{MICRO_ALONE}    
\end{itemize}


%%%%%%%%%%%%%%%%%%%%%%%%%%%%%%%%%%%%%%%%%%%%%%%%%%%%%%%%%%%%%%%%%%%%%%%%%%%%%%%%%%%%


%\begin{alltt}
%<nXSf> = <  \(\nu\sigma_{1}^{f}\) \(\nu\sigma_{2}^{f} \dots \nu\sigma_{g}^{f}\) > 
%\end{alltt}
