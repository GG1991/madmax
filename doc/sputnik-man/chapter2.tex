\chapter*{\sputnik execution}

%%%%%%%%%%%%%%%%%%%%%%%%%%%%%%%%%%%%%%%%%%%%%%%%%%%%%%%%%%%%%%%%%%%%%%%%%%%%%%%%%%%%

\definecolor{mygreen}{rgb}{0.3, 0.9, 0.1}

%%%%%%%%%%%%%%%%%%%%%%%%%%%%%%%%%%%%%%%%%%%%%%%%%%%%%%%%%%%%%%%%%%%%%%%%%%%%%%%%%%%

\section{Code execution}

The \sputnik programs \macro and \micro can be executed in different modes: standalone
execution or coupled. Generally the standalone execution is used for performing tests of 
each of the codes independently for debugging purposes, while the coupled execution is
used for solving a problem by the multi-scale approach.

\subsection{\macro standalone}

Performing a mesh partition:
\begin{lstlisting}[frame=single,language=bash]
<@\textcolor{blue}{mpirun}@> -np <np> <@\textcolor{OliveGreen}{macro}@> -mesh <mesh.msh>
\end{lstlisting}

Testing a \macro calculation with input file (see section for further details), this one should not have materials defined with \micflag.
The calculation involves the solution of PDE.
\begin{lstlisting}[frame=single,language=bash]
<@\textcolor{blue}{mpirun}@> -np <np> <@\textcolor{OliveGreen}{macro}@> -input <input.mac>
\end{lstlisting}

\subsection{\micro standalone}

Performing a mesh partition:
\begin{lstlisting}[frame=single,language=bash]
<@\textcolor{blue}{mpirun}@> -np <np> <@\textcolor{OliveGreen}{micro}@> -mesh <mesh.msh>
\end{lstlisting}

Testing a \micro calculation with input file (see section for further details).
The calculation involves the solution of PDE.
\begin{lstlisting}[frame=single,language=bash]
<@\textcolor{blue}{mpirun}@> -np <np> <@\textcolor{OliveGreen}{micro}@> -input <input.mic>
\end{lstlisting}

\subsection{\macro and \micro coupled}

Performing a multi-scale calculation
\begin{lstlisting}[frame=single,language=bash]
<@\textcolor{blue}{mpirun}@> -np <np_mac> <@\textcolor{OliveGreen}{macro}@> -input <input.mac> :
      -np <np_mic> <@\textcolor{OliveGreen}{micro}@> -input <input.mic>
\end{lstlisting}

%%%%%%%%%%%%%%%%%%%%%%%%%%%%%%%%%%%%%%%%%%%%%%%%%%%%%%%%%%%%%%%%%%%%%%%%%%%%%%%%%%%

\section{Input file}

\subsection{Mesh information}

\begin{Verbatim}[frame=single,commandchars=\\\{\}]

\textcolor{Green}{# MACRO INFORMATION}
\textcolor{Green}{# <# PROCESSES> }
    2  
\textcolor{Green}{# MICRO INFORMATION}
\textcolor{Green}{# <# PROCESSES> }
    6  
\textcolor{Green}{# <#KINDS> }
    3
\textcolor{Green}{# <# PROCESSES KIND 0> <# PROCESSES KIND 1> <# PROCESSES KIND 2>}
    1                    1                    1
\end{Verbatim}


%%%%%%%%%%%%%%%%%%%%%%%%%%%%%%%%%%%%%%%%%%%%%%%%%%%%%%%%%%%%%%%%%%%%%%%%%%%%%%%%%%%%


%\begin{alltt}
%<nXSf> = <  \(\nu\sigma_{1}^{f}\) \(\nu\sigma_{2}^{f} \dots \nu\sigma_{g}^{f}\) > 
%\end{alltt}

