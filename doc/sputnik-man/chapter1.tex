\chapter{Code design}

\section{Introduction}

\begin{tikzpicture}[scale=0.9,node distance = 2cm, auto]

   \tikzstyle{block}  = [rectangle, draw, fill=blue!20 , text width=5em, text centered, rounded corners, minimum height=3em]
   \tikzstyle{blockf} = [rectangle, draw, fill=red!20  , text width=10em, text centered, rounded corners, minimum height=3em]
   \tikzstyle{blockc} = [rectangle, draw, fill=green!20, text width=10em, text centered, rounded corners, minimum height=3em]
   \tikzstyle{line}   = [draw, -latex']
   \tikzstyle{blocke} = [rectangle, draw, text width=9em, text centered, rounded corners, minimum height=2.5em]

   % draw a grid for positioning nodes
   \coordinate (bottom_left) at (0,0);
   \coordinate (top_right) at (22,10);
   \draw [dotted, draw=black, fill=white] (bottom_left) grid  (top_right);

   \node [block] (sput) at (5,9)  {\sputnik};
   \node [block] (macr) at (2,7)  {\macro};
   \node [block] (micr) at (8,7) {\micro};
   \node [blockf] (pars) at (13,7) {spu_parse_comm};
   \node [blockc] (parc) at (19,7) {
     \begin{verbatim}
     $spu_comm
        ...
     $end_spu_comm
     \end{verbatim}
     };

   \node [blockf] (colo) at (13,5) {spu_coloring};
   \node [blockf] (colo) at (13,5) {spu_coloring};
   \node [blockf] (colo) at (13,5) {spu_coloring};
   
   \path [line] (sput) -- (macr);
   \path [line] (sput) -- (micr);
   \path [line] (sput) -- (13,9) -- (pars);
   \path [line] (pars) -- (colo);
   \path [line] (pars) -- (parc);
   %\draw [->,line width=2] (thermo)  to[bend left,outer ysep=1em] node {$\sigma$} (coupvol) ;

\end{tikzpicture}

%%%%%%%%%%%%%%%%%%%%%%%%%%%%%%%%%%%%%%%%%%%%%%%%%%%%%%%%%%%%%

\section{\sputnik Datastructures}

In the code exist an array \texttt{id_vec} that in each position \texttt{i} has the \texttt{id} of the code that is
executing (\texttt{id = MACRO|MICRO}). For obtaining this array an \texttt{MPI_Allgather} operation is performed. Using
this array at the it is possible to check is the options on the input file have sense.


\begin{tikzpicture}[font=\ttfamily,
array/.style={matrix of nodes,nodes={draw, minimum size=7mm, fill=green!30},column sep=-\pgflinewidth, row sep=0.5mm, nodes in empty cells,
row 1/.style={nodes={draw=none, fill=none, minimum size=5mm}},
row 1 column 1/.style={nodes={draw}}}]

    \matrix[array] (array) {
        0 & 1 & 2 & 3 & 4 & 5 & 6 & 7 & 8 & 9\\
        1 & 2 & 2 & 1 & 1 & 2 & 2 & 2 & 2 & 2\\};
    %\node[draw, fill=gray, minimum size=4mm] at (array-2-9) (box) {};

    \begin{scope}[on background layer]
    \fill[green!10] (array-1-1.north west) rectangle (array-1-10.south east);
    \end{scope}

    \draw[<->]([yshift=-3mm]array-2-1.south west) -- node[below] {id_vec length = \#proc} ([yshift=-3mm]array-2-10.south east);

    \draw (array-1-1.north)--++(90:3mm) node [above] (first) {global rank};
    \draw (array-2-10.east)--++(0:3mm) node [right]{IDs = 1(MACRO)|2(MICRO)};
    %\node [align=center, anchor=south] at (array-2-9.north west|-first.south) (8) {IDs\\ MACRO=1 MICRO=1};
    %\draw (8)--(box);
    %
\end{tikzpicture}

%%%%%%%%%%%%%%%%%%%%%%%%%%%%%%%%%%%%%%%%%%%%%%%%%%%%%%%%%%%%%

\section{\sputnik Communication Approaches}

\subsection{\texttt{Scheme 2}}

All the macro-processes communicates with the same micro-processes. In this approach is possible to parallelize the
assembling process doing calculations on both micro-structures at the same time.

\begin{tikzpicture}[->,>=stealth',shorten >=1pt,auto,node distance=3em,semithick]

   \tikzstyle{state} =[fill=red,draw=none,text=white,minimum size=0.1cm]
   \tikzstyle{type2}=[fill=green,draw=none,text=white,minimum size=0.1cm]
   \tikzstyle{type3}=[fill=blue,draw=none,text=white,minimum size=0.1cm]

   % draw a grid for positioning nodes
   \coordinate (bottom_left) at (0,0);
   \coordinate (top_right) at (22,10);
   \draw [dotted, draw=black, fill=white] (bottom_left) grid  (top_right);

    \node[draw,minimum width=35em,minimum height=25em,font=\small,label={[xshift=0.0cm, yshift=-8.0cm]
  	MPI\_COMM\_WORLD} ]  at (8,5) (world){} ;
  
    \node[draw,minimum width=3em,minimum height=16em,font=\small, label=above:{Macro-structure}] 
    at ([xshift=-7em]world.center) (macro){};
  
    \node[draw,minimum width=9em,minimum height=3em,font=\small, label=right:{Micro-structure 1}] 
    at ([xshift=+4em,yshift=+5em]world.center) (micro1){} ;
  
    \node[draw,minimum width=9em,minimum height=3em,font=\small, label=right:{micro-structure 2}] 
    at ([xshift=+4em,yshift=-2em]world.center) (micro2){} ;
  
  
    \node[state] (A) at ([yshift=-2em]macro.north) {};
    \node[state] (B) [below of=A] {};
    \node[state] (C) [below of=B] {};
    \node[state] (D) [below of=C] {};
    \node[state] (E) [below of=D] {};
  
    \node[type2] (f) at ([xshift=+1.5em]micro1.west) {};
    \node[type2] (g) [right of=f] {};
    \node[type2] (h) [right of=g] {};
  
    \node[type3] (i) at ([xshift=+3em]micro2.west) {};
    \node[type3] (j) [right of=i] {};
  
    \draw [<->] (A) -- (B);
    \draw [<->] (B) -- (C);
    \draw [<->] (C) -- (D);
    \draw [<->] (D) -- (E);
  
    \draw [<->] (f) -- (g);
    \draw [<->] (g) -- (h);

    \draw [<->] (i) -- (j);
  
    \path (A) edge node {} (micro1)
          (B) edge node {} (micro1)
          (C) edge node {} (micro1)
          (D) edge node {} (micro1)
          (E) edge node {} (micro1);

    \path (A) edge node {} (micro2)
          (B) edge node {} (micro2)
          (C) edge node {} (micro2)
          (D) edge node {} (micro2)
          (E) edge node {} (micro2);
  
\end{tikzpicture}

%%%%%%%%%%%%%%%%%%%%%%%%%%%%%%%%%%%%%%%%%%%%%%%%%%%%%%%%%%%%%

\subsection{\texttt{Scheme 2}}

All the macro-processes communicates with all the micro-processes. In this approach is possible to parallelize the
assembling process doing calculations on all micro-structures at the same time. The \emph{scheme}

\begin{tikzpicture}[->,>=stealth',shorten >=1pt,auto,node distance=3em,semithick]

   \tikzstyle{state} =[fill=red,draw=none,text=white,minimum size=0.1cm]
   \tikzstyle{type2}=[fill=green,draw=none,text=white,minimum size=0.1cm]
   \tikzstyle{type3}=[fill=blue,draw=none,text=white,minimum size=0.1cm]

   % draw a grid for positioning nodes
   \coordinate (bottom_left) at (0,0);
   \coordinate (top_right) at (22,10);
   \draw [dotted, draw=black, fill=white] (bottom_left) grid  (top_right);

    \node[draw,minimum width=35em,minimum height=25em,font=\small,label={[xshift=0.0cm, yshift=-8.0cm]
  	MPI\_COMM\_WORLD} ]  at (8,5) (world){} ;
  
    \node[draw,minimum width=3em,minimum height=16em,font=\small, label=above:{Macro-structure}] 
    at ([xshift=-7em]world.center) (macro){};
  
    \node[draw,minimum width=9em,minimum height=3em,font=\small, label=right:{Micro-structure 1}] 
    at ([xshift=+4em,yshift=+12em]world.center) (micro1){} ;
  
    \node[draw,minimum width=9em,minimum height=3em,font=\small, label=right:{micro-structure 2}] 
    at ([xshift=+4em,yshift=+4em]world.center) (micro2){} ;

    \node[draw,minimum width=9em,minimum height=3em,font=\small, label=right:{Micro-structure 1}] 
    at ([xshift=+4em,yshift=-4em]world.center) (micro3){} ;
  
    \node[draw,minimum width=9em,minimum height=3em,font=\small, label=right:{micro-structure 2}] 
    at ([xshift=+4em,yshift=-12em]world.center) (micro4){} ;
  
  
    \node[state] (A) at ([yshift=-2em]macro.north) {};
    \node[state] (B) [below of=A] {};
    \node[state] (C) [below of=B] {};
    \node[state] (D) [below of=C] {};
    \node[state] (E) [below of=D] {};
  
    \node[type2] (f) at ([xshift=+1.5em]micro1.west) {};
    \node[type2] (g) [right of=f] {};
    \node[type2] (h) [right of=g] {};
  
    \node[type3] (i) at ([xshift=+3em]micro2.west) {};
    \node[type3] (j) [right of=i] {};
  
    \draw [<->] (A) -- (B);
    \draw [<->] (B) -- (C);
    \draw [<->] (C) -- (D);
    \draw [<->] (D) -- (E);
  
    \draw [<->] (f) -- (g);
    \draw [<->] (g) -- (h);

    \draw [<->] (i) -- (j);
  
    \path (A) edge node {} (micro1)
          (B) edge node {} (micro1)
          (C) edge node {} (micro1)
          (D) edge node {} (micro1)
          (E) edge node {} (micro1);

    \path (A) edge node {} (micro2)
          (B) edge node {} (micro2)
          (C) edge node {} (micro2)
          (D) edge node {} (micro2)
          (E) edge node {} (micro2);

  
    \node[type2] (f2) at ([xshift=+1.5em]micro3.west) {};
    \node[type2] (g2) [right of=f2] {};
    \node[type2] (h2) [right of=g2] {};
  
    \node[type3] (i2) at ([xshift=+3em]micro4.west) {};
    \node[type3] (j2) [right of=i2] {};
  
    \draw [<->] (f2) -- (g2);
    \draw [<->] (g2) -- (h2);

    \draw [<->] (i2) -- (j2);

    \path (A) edge node {} (micro3)
          (B) edge node {} (micro3)
          (C) edge node {} (micro3)
          (D) edge node {} (micro3)
          (E) edge node {} (micro3);

    \path (A) edge node {} (micro4)
          (B) edge node {} (micro4)
          (C) edge node {} (micro4)
          (D) edge node {} (micro4)
          (E) edge node {} (micro4);
  
\end{tikzpicture}

%%%%%%%%%%%%%%%%%%%%%%%%%%%%%%%%%%%%%%%%%%%%%%%%%%%%%%%%%%%%%

\begin{forest}
   for tree={
             font=\ttfamily,
             grow'=0,
             child anchor=west,
             parent anchor=south,
             anchor=west,
             calign=first,
             edge path={
                          \noexpand\path [draw, \forestoption{edge}]
                            (!u.south west) +(7.0pt,0) |- node[fill,inner sep=1.20pt] {} (.child
                                anchor)\forestoption{edge label};
                        },
             before typesetting nodes={
             if n=1
             {insert before={[,phantom]}}
             {}
             },
             fit=band,
             before computing
               xy={l=12pt},
             }
[sputnik
[src
[spu_parser.c]
[list.c]
]
[macro
[src
[mac_parser.c]
[mac_color.c]
[mac_mesh.c]
]
[inc
[macro.h]
]
]
[micro
[src
[mic_parser.c]
[mic_color.c]
[mic_mesh.c]
]
[inc
[micro.h]
]
]
]
\end{forest}

